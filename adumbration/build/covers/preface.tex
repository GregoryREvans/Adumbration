\documentclass[11pt]{article}
\usepackage{fontspec}
\usepackage[utf8]{inputenc}
\setmainfont{STIXGeneral}
\usepackage[paperwidth=8.5in,paperheight=11in,margin=1in,headheight=0.0in,footskip=0.5in,includehead,includefoot,portrait]{geometry}
\usepackage[absolute]{textpos}
\TPGrid[0.5in, 0.25in]{23}{24}
\parindent=0pt
\parskip=12pt
\usepackage{nopageno}
\usepackage{graphicx}
\graphicspath{ {./images/} }
\usepackage{amsmath}
\usepackage{tikz}
\newcommand*\circled[1]{\tikz[baseline=(char.base)]{
            \node[shape=circle,draw,inner sep=1pt] (char) {#1};}}

\begin{document}

\begin{textblock}{23}(0, 1)
\begin{center}
\huge FOREWORD
\end{center}
\end{textblock}

\vspace*{0.15\baselineskip}

\begingroup
\begin{center}
\leftskip0.5in
The word \textit{Adumbration} may be taken to have several meanings. First, a directed shadowing, intentional or otherwise: the shadow which precedes an obect in motion while backlit. It could mean a metaphorical projection of foreshadowing: presage. Also a sketch, an outline, a finished (or unfinished, temporary) text or artwork. Summarily: a skeleton or a premonition outwardly cast.
\rightskip\leftskip
\phantom{text} \hfill (G.R.E.)
\end{center}
\endgroup

\begingroup
\begin{center}
\leftskip0.5in
What happens in the shadow, in the grey regions, also interests us -- all that is elusive and fugitive, all that can be said in those beautiful half tones, or in whispers, in deep shade.
\rightskip\leftskip
\phantom{text} \hfill (The Brothers Quay)
\end{center}
\endgroup

\begingroup
\begin{center}
\leftskip0.5in
What deity in the realms of dementia, what rabid god decocted out of the smoking lobes of hydrophobia could have devised a keeping place for souls so poor as is this flesh. This mawky worm-bent tabernacle.
\rightskip\leftskip
\phantom{text} \hfill (Cormac McCarthy, \textit{Suttree})
\end{center}
\endgroup

\begingroup
\begin{center}
\leftskip0.5in
In the temples of Japan, [...] in the palaces of the nobility and the houses of the common people, what first strikes the eye is the massive roof of tile or thatch and the heavy darkness that hangs beneath the eaves. Even at midday cavernous darkness spreads over all beneath the roof's edge, making entryway, doors, walls, and pillars all but invisible. 
\rightskip\leftskip
\phantom{text} \hfill (Junichiro Tanizaki, \textit{In Praise of Shadows})
\end{center}
\endgroup

\begingroup
\begin{center}
\leftskip0.5in
Sat alone. He, in absentia, to receive those loops: braided shadows of unheard prayers. The clock does not tell her. Thread, she speaks. In knots.
\rightskip\leftskip
\phantom{text} \hfill (G.R.E.)
\end{center}
\endgroup

\begingroup
\begin{center}
\leftskip0.5in
In the square there is the wall where the old men sit and watch the young go by; he is seated in a row with them. Desires are already memories.
\rightskip\leftskip
\phantom{text} \hfill (Italo Calvino, \textit{Invisible Cities})
\end{center}
\endgroup

\begingroup
\begin{center}
\leftskip0.5in
I thought: ``You reach a moment in life when, among the people you have known, the dead outnumber the living. And the mind refuses to accept more faces, more expressions: on every new face you encounter, it prints the old forms, for each one it finds the most suitable mask.''
\rightskip\leftskip
\phantom{text} \hfill (Italo Calvino, \textit{Invisible Cities})
\end{center}
\endgroup

\vspace*{5\baselineskip}

\begin{center}
\huge PERFORMANCE NOTES
\end{center}

\begingroup
\begin{center}

\leftskip0.25in
\pmb{Pitch} : At times throughout the score, justly tuned intervals are indicated by the use of Helmholtz-Ellis notation combined with cent deviations from equal temperament for use with an electronic tuner. When no example pitch is given with the cent deviation, the mark is a deviation of the nearest ``standard'' accidental. If the performers wish to interpret the score without cent-tuning, the approximation of pitches to the nearest semi-tone is acceptable. When Helmholtz-Ellis notation is not given, the pitches are to be played as usual.
\rightskip\leftskip
\phantom{text} \hfill \phantom{()}

\leftskip0.25in
\pmb{Bow Rotation Indications} : \circled{1} ordinario (abbreviated as $ord.$) and \circled{2} $col \ legno \ tratto$ (abbreviated as $clt.$). When these abbreviations are not present, the performer should default to $ordinario$ bowing techniques. \circled{3} In measure 208, the second violin is given the direction to place the bow on the two lowest strings and to twist the bow in a clockwise motion, to create a crackling sound. This is written in degrees of a circle, where the length of the bow is the diameter, with an arrow pointing in which direction the bow should be twisted.
\rightskip\leftskip
\phantom{text} \hfill \phantom{()}

\leftskip0.25in
\pmb{String Contact Points} : The indications of string contact positions such as $sul \ tasto$ (abbreviated as $st.$), $sul \ ponticello$ (abbreviated as $sp.$) etc. should be considered as points along the continuum of the length string. The performer should make an effort to smoothly transition from one position to the next when abbreviations are connected by an arrow. When this arrow is not present, the performer should change position $subito$.
\rightskip\leftskip
\phantom{text} \hfill \phantom{()}

\leftskip0.25in
\pmb{Dynamic Indications} : Dynamics marks should be considered ``effort dynamics.'' As such, $forte$ represents a heavy bow pressure rather than a ``loud'' resultant sound. Likewise, $piano$ represents a light bow pressure as opposed to a ``quiet'' resultant sound. These indications will often result in unusual bowing timbres when combined with the String Contact Points, and finger pressure alterations. These are the desired effects.
\rightskip\leftskip
\phantom{text} \hfill \phantom{()}

\leftskip0.25in
\pmb{Miscellaneous} : \circled{1} Tremoli should be performed as fast as possible and not as a measured subdivision of the duration to which they are attached, with $XFB$ representing an $extremely \ flautando \ bow$. When the $noise$ indication is combined with tremolo, the motion should be extremely tight. $XFB$ tremoli should be slightly irregular, comprised of almost full bow strokes. \circled{2} $XSB$ stands for ``extremely slow bow'' which refers to a bow speed so slow that intermittent clicks occur. \circled{3} Diamond note heads represent a left hand finger pressure of a natural harmonic, \circled{4} while a triangle note head indicates a finger pressure in between harmonic and $normale$, which often results in a multiphonic. \circled{5} Accidentals apply only to the pitch which they immediately precede, but persist through ties. \circled{6} The choice to perform this piece either $senza \ vibrato$ or $con \ vibrato$ is left to the performers except where $sv.$ is required.
\rightskip\leftskip
\phantom{text} \hfill \phantom{()}
\end{center}
\endgroup

\vspace*{17\baselineskip}

\begin{center}
c. 13'
\end{center}

\vspace*{17\baselineskip}

\end{document}
