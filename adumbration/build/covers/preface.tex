\documentclass[11pt]{article}
\usepackage{fontspec}
\usepackage[utf8]{inputenc}
\setmainfont{STIXGeneral}
\usepackage[paperwidth=8.5in,paperheight=11in,margin=1in,headheight=0.0in,footskip=0.5in,includehead,includefoot,portrait]{geometry}
\usepackage[absolute]{textpos}
\TPGrid[0.5in, 0.25in]{23}{24}
\parindent=0pt
\parskip=12pt
\usepackage{nopageno}
\usepackage{graphicx}
\graphicspath{ {./images/} }
\usepackage{amsmath}
\usepackage{tikz}
\newcommand*\circled[1]{\tikz[baseline=(char.base)]{
            \node[shape=circle,draw,inner sep=1pt] (char) {#1};}}

\begin{document}

\begin{textblock}{23}(0, 1)
\begin{center}
\huge FOREWORD
\end{center}
\end{textblock}

\vspace*{0.25\baselineskip}

\begingroup
\begin{center}
\leftskip1.25in
The word \textit{Adumbration} may be taken to have several meanings. First, a directed shadowing. The shadow which precedes an obect in motion while backlit. It could mean a metaphorical projection of foreshadowing. Also a sketch, an outline, a finished (or unfinished, temporary) text or artwork.
\rightskip\leftskip
\phantom{text} \hfill (G.R.E.)
\end{center}
\endgroup

\vspace*{1.25\baselineskip}

\begin{center}
\huge PERFORMANCE NOTES
\end{center}

\begingroup
\begin{center}
\leftskip0.25in
\pmb{Bow Rotation Indications} : \circled{1} ordinario (abbreviated as $ord.$) and \circled{2} $col \ legno \ tratto$ (abbreviated as $clt.$). When these abbreviations are not present, the performer should default to $ordinario$ bowing techniques.
\rightskip\leftskip
\phantom{text} \hfill \phantom{()}

\leftskip0.25in
\pmb{String Contact Points} : The indications of string contact positions such as $sul \ tasto$ (abbreviated as $st.$), $sul \ ponticello$ (abbreviated as $sp.$) etc. should be considered as points along the continuum of the length string. The performer should make an effort to smoothly transition from one position to the next when abbreviations are connected by an arrow. When this arrow is not present, the performer should change position $subito$.
\rightskip\leftskip
\phantom{text} \hfill \phantom{()}

\leftskip0.25in
\pmb{Dynamic Indications} : Dynamics marks should be considered ``effort dynamics.'' As such, $forte$ represents a heavy bow pressure rather than a ``loud'' resultant sound. Likewise, $piano$ represents a light bow pressure as opposed to a ``quiet'' resultant sound. These indications will often result in unusual bowing timbres when combined with the String Contact Points, and finger pressure alterations. These are the desired effects.
\rightskip\leftskip
\phantom{text} \hfill \phantom{()}

\leftskip0.25in
\pmb{Miscellaneous} : \circled{1} Tremoli should be performed as fast as possible and not as a measured subdivision of the duration to which they are attached, with abbreviations $IT$, $TO$, and $XFB$ representing $irregular \ tremolo$, $tremolo \ ordinario$, and $extreme \ flautando \ bow$ respectively. $XFB$ tremoli should be slightly irregular (less than $IT$), comprised of almost full bow strokes. \circled{2} Diamond note heads represent a left hand finger pressure of a natural harmonic, \circled{3} while a triangle note head indicates a finger pressure in between harmonic and $normale$, which often results in a multiphonic. \circled{4} Accidentals apply only to the pitch which they immediately precede, but persist through ties. \circled{5} The choice to perform this piece either $senza \ vibrato$ or $con \ vibrato$ is left to the performers except where $sv.$ is required.
\rightskip\leftskip
\phantom{text} \hfill \phantom{()}
\end{center}
\endgroup

\vspace*{17\baselineskip}

\begin{center}
c. 3'50''
\end{center}

\vspace*{27\baselineskip}

\end{document}
